\documentclass{article}
\usepackage{amssymb}
\usepackage[margin=0.8in]{geometry}
\usepackage{hyperref}
\usepackage{tikz}
\usepackage{caption}
\usepackage{subcaption}
\newtheorem{theorem}{Theorem}[section]
\newtheorem{lemma}[theorem]{Lemma}
\newtheorem{proposition}[theorem]{Proposition}
\newtheorem{corollary}[theorem]{Corollary}
\newtheorem{definition}[theorem]{Definition}
\newtheorem{remark}[theorem]{Remark}
\newtheorem{property}[theorem]{Property}
\pagestyle{plain}

\usepackage{datetime}
\newdate{date}{22}{04}{2024}

\begin{document}
\title{Extremal Graph Theory - Financial Risk Assessment}
\author{Blake Marterella}
\date{\displaydate{date}}

\maketitle

\section*{Abstract}
    Summarizing your report in a short paragraph.
    Hello World!

\tableofcontents

\section{Introduction}

Briefly introduce the topic.

\subsection{Background}

\subsection{Why Is This Of Interest?}

\subsection{Motivation}

\section{Main Section}

Relevant definitions, theorems, examples, etc. In between write down your analysis cohesively.

\begin{definition} State anything not defined in class here.
\end{definition}
By \footnote{footnote} we find inequality
You reference or cite by using labels. E.g. By Theorem \ref{p1} we find..., or from \cite{Y10} we know...\\
\\
Itemized list:
\begin{itemize}
\item Every nonzero real number has a reciprocal.
\item There is a real number with no reciprocal.
\end{itemize}
Table:
\begin{center}
\begin{tabular}{|c|c|c|c|c|} 
 \hline
 $p$ & $q$ & $\neg p$ & $\neg p\lor q$ & $p\to q$ \\ 
 \hline
 T & T & F & T & T \\
 T & F & F & F & F \\
 F & T & T & T & T \\
 F & F & T & T & T \\
 \hline
\end{tabular}
\end{center}
Equation array:
\begin{eqnarray*}
31&=&17+14\\
17&=&14+3\\
14&=&4(3)+2\\
3=2+1
\end{eqnarray*}
Implication sign:
$$x>2\Longrightarrow x^2>4$$
Other typical symbols include $\to$, $\equiv$, $\neg$, $\land$, and $\lor$.\\

Where appropriate you can put words in \textbf{boldface} or \textit{italics}, or \underline{underlined}. However, different colors like \textcolor{blue}{blue} should be avoided in a paper (unless it is really necessary).

\section{Conclusion}

Based on your research, write down what you discovered. In particular, discuss related areas of interest and any potential directions for future investigation.

\bibliographystyle{alpha}
\begin{thebibliography}{99}
\bibitem[ANHF11]{ANHF}
M. J. Ablowitz, S. D. Nixon, T. P. Horikis, and D. J. Frantzeskakis, \emph{Perturbations of dark solitons}, Proc. R. Soc. A Vol \textbf{467} (2011), 2597-2621.
\bibitem[HN98]{HY}
N. Hayashi and P. I. Naumkin, \emph{Asymptotics for Large Time of Solutions to the Nonlinear Schr\"odinger and Hartree Equations}, American Journal of Mathematics, Vol \textbf{120} No.2 (1998) 369-389.
\bibitem[Y10]{Y10}
J. Yang, \emph{Nonlinear Waves in Integrable and Nonintegrable Systems}, SIAM, Philadelphia (2010).
\bibitem[Author initials and year]{Z}
Authors, \emph{Title of Book or Paper}, Journal, Volume \textbf{Number}, Publisher (Year), page numbers.
\end{thebibliography}

\end{document}