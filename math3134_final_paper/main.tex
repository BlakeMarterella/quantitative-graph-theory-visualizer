\documentclass{article}
\usepackage{amssymb}
\usepackage[margin=0.8in]{geometry}
\usepackage{hyperref}
\usepackage{tikz}
\usepackage{caption}
\usepackage{subcaption}
\usepackage{bookmark}
\newtheorem{theorem}{Theorem}[section]
\newtheorem{lemma}[theorem]{Lemma}
\newtheorem{proposition}[theorem]{Proposition}
\newtheorem{corollary}[theorem]{Corollary}
\newtheorem{definition}[theorem]{Definition}
\newtheorem{remark}[theorem]{Remark}
\newtheorem{property}[theorem]{Property}
\pagestyle{plain}

\usepackage{datetime}
\newdate{date}{22}{04}{2024}

\begin{document}
\title{Extremal Graph Theory - Financial Risk Assessment}
\author{Blake Marterella}
\date{\displaydate{date}}

\maketitle

\section*{Abstract}
The goal of this paper is to provide:
Visualizations of trades over time that enforce the concepts about diversification and risk mitigation explored in the paper
Give a better understanding on where individuals can begin to get into stock trading without the worry (maybe start with the SP 500 and choose stocks from there?)

\tableofcontents

% [Introduction]
\section{Introduction}

\subsection{Interest}

Explain why this interection of fiance and combinatorics is crucial for risk management and investment strategies.

\subsection{Brief History}

Introdfuce the evolution of graph theory applications in financial markets. 

\subsection{Motivation}

Discuss why modern porfolio theory benefits from advanced mathematical tools such as combinatorics and graph theory.

% Let this flow into the definitions section

% [Main Section]
\section{Background}

This is the first part of the main body of the paper. Here you will define the key concepts and terms that will be used throughout the paper.

\subsection{Definitions}

% Extremal Graph Theory

\begin{theorem}[Extremal Graph Theorem]
    Let $G$ be a graph with $n$ vertices and $m$ edges. Then, if $G$ does not contain a subgraph isomorphic to $K_{r+1}$, the complete graph on $r+1$ vertices, then $m \leq \frac{r}{2}(n-1)$.
\end{theorem}

% Spearman Rank Coefficent

\begin{definition}[Spearman Rank Coefficent]
\end{definition}

\[
\rho = 1 - \frac{6 \sum d_i^2}{n(n^2-1)}
\]

Go into detail about the spearman rank coefficent

% Greedy Coloring Algorithm

\begin{definition}[Welsch-Powwell Algorithm]
\end{definition}

Go into detail about the coloring alogirthm



Briefly define graph theory terms that will be used (vertices, edges, etc.). 


Convert the concept of a portfolio from a spreadsheet to a graph with vertices and edges. This concept is the central point of the paper.


% [Main Section]
\section{Portfolio Optimization}

Optimization and Diversification
Extremal Graph Theory

\begin{itemize}
    \item Theoretical Framework: Explain the extremal graph theorem.
    \item Application: Demonstrate how this theorem can predict the maximum or minimum number of edges under certain conditions, which translates to understanding the limits of diversification in a portfolio.
    \item Examples: Provide hypothetical examples of portfolios and how the theorem applies.
\end{itemize}


% [Main Section]
\section{Risk Assessment}
Coloring algorithms for risk assesment and management

\begin{itemize}
    \item Concept Introduction: Explain what graph coloring is and the significance of using different colors.
    \item Implementation: How coloring can be used to represent different levels of risk or different asset classes.
    \item Practical Example: A case study where coloring helps in decision-making about asset allocation or identifying over-concentrated sectors
\end{itemize}


% [Main Section]
\section{Holding Vizualization}

Correlation Graphs for Portfolio Holdings

\begin{itemize}
    \item Graph Construction: Discuss how to build a graph where vertices represent assets and edges represent correlations between returns.
    \item Analysis Techniques: Use threshold levels to add/remove edges or use weights to show the strength of correlations.
    \item Visualization: Include a section on how these graphs can visually represent portfolio diversification and the interconnections between assets.
\end{itemize}


\section{Conclusion}

\begin{itemize}
    \item Summary: Recap how graph theory enhances portfolio management.
    \item Future Directions: Suggest how further research could integrate other combinatorial techniques or advanced graph theory concepts.
    \item Open Problems: Pose any unresolved questions or potential for new research that your paper hints at.
\end{itemize}


\bibliographystyle{alpha}
\begin{thebibliography}{99}
    \bibitem[ANHF11]{ANHF}
    M. J. Ablowitz, S. D. Nixon, T. P. Horikis, and D. J. Frantzeskakis, \emph{Perturbations of dark solitons}, Proc. R. Soc. A Vol \textbf{467} (2011), 2597-2621.
    \bibitem[Author initials and year]{Z}
    Authors, \emph{Title of Book or Paper}, Journal, Volume \textbf{Number}, Publisher (Year), page numbers.
\end{thebibliography}

\end{document}

% Unused starter code

% \begin{definition} State anything not defined in class here.
% \end{definition}
% By \footnote{footnote} we find inequality
% You reference or cite by using labels. E.g. By Theorem \ref{p1} we find..., or from \cite{Y10} we know...\\
% \\
% Itemized list:
% \begin{itemize}
% \item Every nonzero real number has a reciprocal.
% \item There is a real number with no reciprocal.
% \end{itemize}
% Table:
% \begin{center}
% \begin{tabular}{|c|c|c|c|c|} 
%  \hline
%  $p$ & $q$ & $\neg p$ & $\neg p\lor q$ & $p\to q$ \\ 
%  \hline
%  T & T & F & T & T \\
%  T & F & F & F & F \\
%  F & T & T & T & T \\
%  F & F & T & T & T \\
%  \hline
% \end{tabular}
% \end{center}
% Equation array:
% \begin{eqnarray*}
% 31&=&17+14\\
% 17&=&14+3\\
% 14&=&4(3)+2\\
% 3=2+1
% \end{eqnarray*}
% Implication sign:
% $$x>2\Longrightarrow x^2>4$$
% Other typical symbols include $\to$, $\equiv$, $\neg$, $\land$, and $\lor$.\\

% Where appropriate you can put words in \textbf{boldface} or \textit{italics}, or \underline{underlined}. However, different colors like \textcolor{blue}{blue} should be avoided in a paper (unless it is really necessary).